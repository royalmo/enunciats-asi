\documentclass{practicaitic}
\usepackage{parskip}
\usepackage{graphicx}

\usepackage[onelevel]{exercici}
%\usepackage[utf8]{inputenc}

\usepackage{tcolorbox}

\numpract{3}
\title{Pràctica 4: Servei de correu}

\assignatura{Aplicacions i Serveis sobre Internet}
\autor{Eric Roy Almonacid \and Francisco del Águila López}

\begin{document}

\section{Introducció}

Per poder realitzar aquesta pràctica és indispensable haver completat la
pràctica 1 i 3, donat que treballarem sobre el VPS i domini creats anteriorment.

\begin{tcolorbox}[
  title=Atenció,
  colback=red!10, colframe=red!50,
  rounded corners
]
Per aquesta pràctica el servidor hauria de poder enviar paquets TCP al 
port 25. Si utilitzeu \textit{Hetzner}, haureu de realitzar una petició
per desbloquejar els ports a la secció de \textit{Limits}.
La documentació de \textit{Hetzner} demana que els usuaris tinguin una
antiguitat superior a un mes per treure els límits.
\newline

Si no aconseguiu desbloquejar l'accés a aquests ports dirigiu-vos a
l'apartat \ref{sec:plan-b}.
\end{tcolorbox}

\subsection{Objectius}

Al finalitzar aquesta pràctica, l'estudiantat haurà:
\begin{enumerate}
  \item Entès el funcionament i l'arquitectura del servei de correu.
  \item Utilitzat \texttt{postfix} per enviar correus de forma local.
  \item Configurat un servidor de recepció de correus obert a Internet.
  \item Configurat l'enviament de correus a Internet des del servidor.
  \item Entès totes les tasques addicionals que cal dur a terme per
  esdevenir un remitent de correu fiable.
\end{enumerate}

\subsection{Condicions}

Aqeusta pràctica està cal·librada per ésser treballada en equips de 2 persones,
i té una durada de 2 setmanes.

\subsection{Lliuraments}

S'haurà de realitzar una entrega a Atenea i mantenir operatiu el servidor 
i domini creats fins que la pràctica sigui avaluada. El format de l'entrega
està detallat a l'apartat \ref{sec:entrega}.

\section{El correu electrònic}

El correu electrònic que coneixem avui es segueix basant en la
implementació original que va aparéixer a ARPANET l'any
1981, quan es va publicar les especificacions del protocol SMTP
(\textit{Simple Mail Transfer Protocol}). Aleshores, Internet consistia
en una xarxa amb molt pocs nodes i de difícil accés.

L'extensió d'Internet i la facilitat d'enviar correus massivament o bé
impersonant algú altre han obligat crear extensions de seguretat per
SMTP: des d'utilitzar TLS fins a autenticar els missatges en funció
de registres en el DNS. Hi ha molts factors a tenir presents si
es vol enviar correus electrònics a grans proveïdors com Gmail.
Tot plegat fa que ens puguem preguntar si la S de SMTP segueix
significant \textit{Simple}.

Aquesta pràctica pretén oferir una guia per poder convertir el vostre
VPS en un servidor de correu i així no haver de pagar mensualitats 
massa elevades per només disposar d'una adreça electrònica corporativa.
Hi ha projectes com \href{https://docs.mailway.app/self-host/}{MailWay} o 
\href{https://docs.mailcow.email}{MailCow} que permeten configurar un
servidor de correu en pocs passos, però en aquesta pràctica utilitzarem
\texttt{postfix} per entendre realment què estem fent.

\subsection{La visió original}

Al ser un protocol que porta més de 40 anys en funcionament, és d'esperar
que SMTP tingui peculiaritats degudes als sistemes d'aquella època. Per
entendre el perquè d'algunes funcionalitats que veurem més endavant és
convenient dedicar uns paràgrafs al passat d'aquest sistema.

En el moment de la seva creació, la gran majoria d'usuaris només disposaven
d'un únic ordinador, sovint compartit amb altres persones, i només accedia
al correu a través d'aquest. Per tant, era lògic pensar que es podria
utilitzar la notació \texttt{usuari@maquina} com a identificador
per enviar correus. Si es vol una altra bústia, només cal crear un
nou usuari al sistema.

A més a més, cada ordinador tenia una adreça IP pública. Només calia
fer coincidir el \textit{hostname} de la màquina amb el nom de domini
d'aquesta i ja es disposaria d'un identificador accesible des de tot
arreu. Per això veure que \texttt{postfix} posa complicacions si volem
gesionar diversos noms de domini des d'un mateix servidor.

Finalment, cal tenir present que no tots els ordinadors estaven
encesos i encara menys connectats a internet tota l'estona (sobretot
quan les connexions amb ADSL es facturaven per temps d'utilització).
Així doncs, és possible que un usuari envii un correu electrònic a
una màquina que està apagada, i que estigui desconnectat quan l'altre
el pugui rebre. Per aquest motiu es van dissenyar els servidors MTA
(\textit{Mail Transfer Agent}), que rebien el correu dels usuaris o
MUA (\textit{Mail User Agent}) i s'encarregaven de dur-lo al
destinatari i esperar si feia falta.

Actualment, tot i seguir podent utilitzar servidors MTA com si res,
gairebé tots els servidors de correu estan sempre disponibles, pel
que han perdut la seva utilitat.

\subsection{Enviament en local}

El correu més bàsic que es pot enviar és entre dos usuaris de la
mateixa màquina. En aquest apartat configurarem el servidor
per poder enviar-vos correus entre els diferents integrants
del grup.

\begin{tasca}
  TODO install postfix, localhost, try, send mail to profe
\end{tasca}

\subsection{Recepció des d'Internet}

\begin{previ}
  TODO posar registre MX al DNS. Es poden posar múltiples,
  perquè si igualment SMTP funcionava amb downtimes?
\end{previ}

\begin{tasca}
  Enviar mail des de l'adreça UPC al vps, un per cada usuari.
\end{tasca}

\subsection{Us d'alies}

\begin{tasca}
  Posar esutdiantat@ que redirigeixi als dos estudiants, tots@ també a profe. Enviar mail.
\end{tasca}

\subsection{Enviament a través d'Internet}

TODO Potser posar el warning aquí?

\begin{previ}
  Comprovar que hi hagi els ports

  nc -vv smtp.google.com 25
  nc -vv smtp.google.com 465
\end{previ}

\begin{tasca}
  Envieu mails a altres grups. Podeu inclús parlar-vos amb grups
  que no tinguin adreça pública, sempre i quan estigueu connectats
  a la VPN i tinguin associat un nom de domini a l'IP pertinent.

  Fixeu-vos que postfix
  i exim son dues implementacions diferents del protocol i s'entenen.
\end{tasca}

\begin{tasca}
  envieu un mail al vostre compte UPC (que gestiona Google). Veureu que apareix a SPAM.
  Si no apareix, mireu la bústia de correu de postfix per si
  Google us ha retornat un missatge d'error.
\end{tasca}

\subsubsection{Alternativa per si no es pot utilitzar els ports de SMTP}
\label{sec:plan-b}

Degut a l'abús que fan els criminals dels VPS per enviar correus
fraudulents, els proveïdors solen filtrar les connexions sortints
als ports de SMTP. Moltes vegades és suficient amb un missatge per
eliminar aquest límit, però en alguns casos aquests no s'aconsegueix.

Si vulguéssiu utilitzar el correu de forma comercial no tindríeu cap
altre remei que esperar o canviar de proveïdor, però els terminis
d'aquesta pràctica obliguen a trobar formes alternatives d'experimentar
amb SMTP.

Així doncs, pel que queda d'enunciat de la pràctica, si no podeu
enviar correus a través d'Internet amb els ports de SMTP, ...

\section{Guanyar-se la confiança dels gegants}
\subsection{FQDN}
\subsection{Blacklists}
\subsection{SPF}
\subsection{DKIM}
https://fornex.com/help/dkim-exim/
\subsection{DMARC}

\section{Extensions}
\subsection{Enviar a través d'un client}
\subsection{Rebre a un client (POP/IMAP)}
\subsection{Múltiples dominis}

\section{Avaluació}
\subsection{Entrega}
\label{sec:entrega}

S'ha d'entregar un fitxer comprimit (\texttt{P4\_GX.zip}, on X és el número o 
lletra del grup) amb un fitxer de text amb el següent format:

\begin{verbatim}
Grup: GX
VPS: srv1.asi.itic.cat
Contrasenya usuari profe: Lkv5YYyON8X
Port SSH: 22

(Comentaris referents a l'entrega, si s'escau)
\end{verbatim}

La majoria de dades son referents a les pràctiques anteriors. Es demana
que les torneu a escriure, així teniu la possibiltat de canviar-les.

Al servidor, creeu el directori \texttt{/entregues} a l'arrel, i a dins un
directori \texttt{/entregues/p4}. Allà hi heu de deixar els scripts i/o manuals
(el que cregueu convenient) que us farien falta si mai haguéssiu de repetir
aquesta pràctica. Afegiu aquests fitxers també al fitxer comprimit de l'entrega.

Per exemple, podríeu indicar:
\begin{itemize}
  \item aaaa
\end{itemize}

No ha de ser extens; aquests documents us serviran a vosaltres si mai heu de
tornar a realitzar aquest procés.

\subsection{Qualificació}

Aquesta pràctica s'avaluarà de la següent manera. La puntuació màxima és 100.

\begin{center}%{|p{1cm}|p{3cm}|}
  \begin{tabular}{p{0.7\linewidth} l}
  \hline
  Concepte & Rang \\ \hline
  Fitxer de l'entrega (\texttt{P4\_GX.zip}) amb format correcte & $[-20, 0]$ \\
  aaaa & $[0, 20]$ \\
  aaaa & $[0, 10]$ \\
  aaaa & $[0, 10]$ \\
  Caaa & $[0, 20]$ \\
  Caaa & $[0, 10]$ \\
  Qualitat dels scripts/tutorials de \texttt{/entregues/p4} & $[0,30]$ \\
  El servidor presenta problemes de seguretat greus & $[-10,0]$ \\
  \hline
  \end{tabular}
\end{center}

Tingueu present que si algun element de la taula anterior no es pot avaluar
aquest es qualificarà amb la nota més baixa.

Si es detecta algun tipus de frau en l'entrega aquesta rebrà una puntuació de zero.

\end{document}
