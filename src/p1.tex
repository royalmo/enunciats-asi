\documentclass{practicaitic}
\usepackage{lst-vhdl, lst-shell}
\usepackage{parskip}

\numpract{1}
\title{Pràctica 1: Posada en marxa d'un VPS}

\assignatura{Aplicacions i Serveis sobre Internet}
\autor{Eric Roy Almonacid \and Francisco del Águila López}

\begin{document}

\section{Introducció}

Aquesta pràctica és l'inici d'un seguit de pràctiques seqüencials, on
l'objectiu final és disposar d'un servidor obert a internet i el coneixement
necessari per poder-hi configurar gairebé tots els serveis \textit{self-hosted}
que hi ha disponibles al mercat.

\subsection{Objectius}

Al finalitzar aquesta pràctica, l'estudiantat haurà:
\begin{enumerate}
  \item Creat un servidor en local o amb un proveïdor de serveis al núvol.
  \item Configurat l'adreçament IPv4 i IPv6 al servidor.
  \item Creats diferents usuaris al servidor amb grups diferents.
  \item Configurat SSH per clau privada.
\end{enumerate}

\subsection{Condicions}

Aqeusta pràctica està cal·librada per ésser treballada en equips de 2 persones,
i té una durada de 2 setmanes.

\subsection{Lliuraments}

S'haurà de realitzar una entrega a Atenea i mantenir operatiu el servidor creat
fins que la pràctica sigui avaluada. El format de la entrega està detallat a
l'apartat \ref{sec:entrega}.

\section{El concepte de VPS}

Hi ha diverses alternatives per poder disposar dels seus propis serveis a
Internet:

\begin{itemize}
  \item Muntar un servidor a casa seva: és una alternativa més econòmica, però
  requereix tenir un ordinador sempre connectat a l'encaminador i una IP pública.
  Actualment molts proveïdors d'internet col·loquen els seus usuaris darrere un
  \textit{Carrier-Grade NAT} (CG-NAT), que els impedeix configurar la
  redirecció de ports de la seva adreça IP pública, ja que és compartida amb la
  resta d'usuaris.
  \item Llogar recursos a tercers: la alternativa més senzilla, ja no ens hem
  de preocupar de tota la part del maquinari.
\end{itemize}

Tanmateix, llogar un servidor sencer no és sempre necessari: sovint amb pocs
recursos ja es pot desplegar serveis com \textit{Moodle}, \textit{Wordpress},
o fins i tot un servidor de \textit{Minecraft}.
Així doncs, per abaratir els preus es poden crear diferents màquines virtuals
en un servidor i llogar cadascuna per separat.

Un \textit{Virtual Private Server} (VPS) és una d'aquestes màquines, amb una
adreça única assignada per fer-la accessible des d'internet. En aquesta
pràctica llogarem i configurarem el nostre primer VPS. Veureu que, a efectes pràctics,
no sabrem ni que estem treballant sobre una màquina virtual.

\section{Desenvolupament de la pràctica}

La majoria de proveïdors
ofereixen serveis a preus molt baixos i amb períodes de prova molt amplis, però
si ho preferiu podeu realitzar les pràctiques amb un ordinador de casa vostra.

En aquests casos haureu de tenir present els requeriments d'adreça IP i
\textit{uptime} (haurà d'estar sempre accessible), i haureu d'adaptar els
enunciats pel vostre cas particular.

\subsection{Creació del VPS}

TODO AFEGIR CAPTURES

Per l'enunciat d'aquesta pràctica hem decidit utilitzar com a exemple el
proveïdor \textit{Hetzner}, però qualsevol que ofereixi un VPS amb IPv4 i
IPv6 públiques haurien de servir (OVH, DigitalOcean, ...).

En el cas de \textit{Hetzner}, es poden compartir enllaços d'afiliat o trobar
codis de descompte a la seva pàgina web per gaudir de 3 mesos de prova. Per
exemple, a la part dreta de qualsevol article de community.hetzner.com hi ha un
codi de descompte.

TASCA 1: Creeu-vos un compte i verifiqueu-lo introduint un mètode de pagament.

TASCA 2: Creeu un nou projecte, i llavors un nou VPS. Com a imatge base us
recomanem utilitzar la darrera versió de Debian. Agafeu l'arquitectura i
recursos més econòmics, i la majoria d'opcions poden ser per defecte.

\subsubsection{Accedir al servidor}

En la majoria de VPS podreu preconfigurar bastantes coses (usuaris, servei SSH),
però partirem des de zero. Un cop creat el servidor accedirem a la consola,
on veurem que només hi ha un usuari inicial (root).

TASCA 3: Accedir a la consola del servidor i comproveu que podeu interactuar amb
ell.

TODO: parlar de firewall de hetzner

TASCA 4: Configureu, si podeu, el tallafocs del vostre proveïdor per acceptar només
els paquets ICMP (ping), i de SSH. Si l'activeu, haureu de pensar més endavant en
afegir excepcions pels servidors que anirem creant.

\subsection{Creació d'usuaris}

Al tenir un servidor obert a Internet és crucial protegir-lo des d'un primer
moment. El primer pas que farem serà crear usuaris per no treballar sempre
amb \texttt{root}.

TASCA 5: Creeu tres nous usuaris al servidor:
\begin{itemize}
  \item Un per cada integrant del grup. Per exemple \texttt{eric} i \texttt{paco}.
  \item Un altre anomenat \texttt{profe} amb el que els professors es connectaran
  per revisar el servidor.
\end{itemize}
Afegiu al grup \texttt{sudo} als tres usuaris, i creeu una contrasenya per
cada un d'ells.

\subsection{Configuració de SSH}

Com us podeu imaginar, no és molt pràctic treballar al servidor des de la
interfície del navegador. SSH o \textit{Secure SHell} és un protocol que permet
accedir a la consola d'un ordinador remotament. Anem a configurar-lo pel nostre
servidor. Ncessitareu el paquet \texttt{openssh-server} al servidor, i \texttt{ssh}
al vostre ordinador (tot i que segurament ja el teniu instal·lat).

TASCA 6: Configureu SSH en el vostre servidor per no acceptar login amb
l'usuari root i engegueu el servidor.

Fent \texttt{ssh -p port usuari@ip\_del\_servidor} des del vostre ordinador 
hauríeu de poder connectar-vos al servidor remotament. El port es pot ometre si
és el per defecte (22), i l'usuari es pot ometre si és el mateix que el del
vostre ordinador.

\subsubsection{Accés amb claus asimètriques}

Actualment, l'únic que protegeix el vostre servidor d'un atac és la contrasenya
del vostre usuari. Això és vulnerable a atacs de força bruta. SSH ofereix com a
alternativa l'autentificació per claus asimètriques.

TASCA 7: Genereu un parell de claus asimètriques. Cada integrant del grup ha de
generar les seves.

TASCA 8: Copieu la clau pública de cada integrant al fitxer
\texttt{~/.ssh/authorized\_keys}. Cada usuari té un fitxer diferent.

TASCA 9: Deshabiliteu les connexions SSH per contrasenya i reinicieu openssh.
Ara no hauríeu de poder connectar-vos des de l'ordinador d'un altre company
de classe (que no té la vostra clau privada), però sí des del vostre.

Finalment, es donarà accés al servidor als professors. A Atenea trobareu penjada
una clau pública que haureu d'afegir a l'usuari \texttt{profe}. Els professors
utilitzaran aquest accés només per avaluar les pràctiques, i no modificaran
l'estat del servidor.

\section{Entrega}
\label{sec:entrega}

S'ha d'entregar un fitxer de text (\texttt{P1\_GX.txt}, on X és el número o 
lletra del grup) amb el següent format:

\begin{verbatim}
Grup: GX
IPv4: 128.140.55.126
IPv6: 2a01:4f8:c013:7f5::1
Contrasenya usuari profe: Lkv5YYyON8X
Port SSH: 22

(Comentaris referents a l'entrega, si s'escau)
\end{verbatim}

Evidentment, heu de posar els valors del vostre servidor i grup. Si feu
servir el port SSH per defecte podeu ometre la darrera línia.

Al servidor, creeu el directori \texttt{/entregues} a l'arrel, i a dins un
directori \texttt{/entregues/p1}. Allà hi heu de deixar els scripts i/o manuals
(el que cregueu convenient) que us farien falta si mai haguéssiu de repetir
aquesta pràctica.

Per exemple, podríeu indicar:
\begin{itemize}
  \item El proveïdor que heu utilitzat, els seus preus i ofertes, i el perquè l'heu escollit.
  \item Solucions a problemes que us heu trobat, i si cal, enllaços a fòrums o manuals.
  \item Comandes que heu utilitzat amb una petita explicació de què fan (1 línia sol ser suficient).
  \item Altres coses que hagueu fet (per exemple, configurar el tallafocs de \textit{Hetzner}).
\end{itemize}

No ha de ser extens; aquests documents us serviran a vosaltres si mai heu de
crear un servidor nou.

\section{Qualificació}

Aquesta pràctica s'avaluarà de la següent manera. La puntuació màxima és 100.

\begin{center}
  \begin{tabular}{ll}
  \hline
  Concepte & Rang \\ \hline
  Fitxer de l'entrega (\texttt{P1\_GX.txt}) amb format correcte & $[-20, 0]$ \\
  Es pot fer ping al servidor per IPv4 & $[0, 5]$ \\
  Es pot fer ping al servidor per IPv6 & $[0, 5]$ \\
  SSH amb contrasenya o amb root habilitat & $[-10, 0]$ \\
  El servidor té usuaris per cada integrant i usuari \texttt{profe} & $[0, 10]$ \\
  Es pot fer SSH al servidor amb la clau privada de \texttt{profe} & $[0, 40]$ \\
  L'usuari \texttt{profe} no pot fer \texttt{sudo} & $[-10,0]$ \\
  Qualitat dels scripts/tutorials de \texttt{/entregues/p1} & $[0,40]$ \\
  El servidor presenta problemes de seguretat greus & $[-10,0]$ \\
  \hline
  \end{tabular}
\end{center}

Nota: si el vostre proveïdor no us permet tenir una adreça IPv6 pública
indiqueu-ho als comentaris del fitxer de la entrega. En aquest cas, l'apartat
sobre IPv4 valdrà el doble i el de IPv6 queda a zero.

Tingueu present que si algun element de la taula anterior no es pot avaluar
(per exemple, al no poder fer \texttt{sudo} no es pot llegir els continguts de \texttt{/entregues/p1})
aquest es qualificarà amb la nota més baixa.

Si es detecta algun tipus de frau en l'entrega aquesta rebrà una puntuació de zero.

\end{document}
