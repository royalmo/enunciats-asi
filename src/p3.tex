\documentclass{practicaitic}
\usepackage{parskip}

\numpract{3}
\title{Pràctica 3: Servei de noms (DNS)}

\assignatura{Aplicacions i Serveis sobre Internet}
\autor{Eric Roy Almonacid \and Francisco del Águila López}

\begin{document}

\section{Introducció}

Per poder realitzar aquesta pràctica és indispensable haver completat la
pràctica 1, donat que treballarem sobre el VPS creat anteriorment.

\subsection{Objectius}

Al finalitzar aquesta pràctica, l'estudiantat haurà:
\begin{enumerate}
  \item Entès el funcionament de la resolució de noms de domini.
  \item Adquirit un domini o subdomini, en el qual en té autoritat.
  \item Creat registres al domini utilitzant un servei de tercers.
  \item Entès el funcionament i configurat DNSSEC al domini.
  \item Entès i configurat DDNS utilitzant serveis de tercers.
\end{enumerate}

\subsection{Condicions}

Aqeusta pràctica està cal·librada per ésser treballada en equips de 2 persones,
i té una durada de 1 a 2 setmanes.

\subsection{Lliuraments}

S'haurà de realitzar una entrega a Atenea i mantenir operatiu el servidor 
i domini creats fins que la pràctica sigui avaluada. El format de la entrega està detallat a
l'apartat \ref{sec:entrega}.

\section{El Servei de Noms de Domini}

El Servei de noms de domini o \textit{Domain Name Server} (DNS) va ser introduït
l'any 1983 amb dues finalitats:

\begin{itemize}
  \item Poder resoldre un nom de domini a una adreça IP, per evitar haver de
  memoritzar-la.
  \item Establir l'arbre de noms: definir l'estructura jeràrquica de noms distribuida
  en zones d'autoritat. Per exemple, qualsevol domini que acabi amb \texttt{upc.edu}
  significarà que pertany a la UPC. Qualsevol domini que acabi amb \texttt{epsem.upc.edu}
  significarà que pertany a l'escola de Manresa.
\end{itemize}

Així doncs, un servidor de DNS es pot distingir segons la funció que realitza:

\begin{itemize}
  \item Autoritatiu: (\textit{authoritative name server}): forma part de l'arbre
    de noms de domini, i defineix
    els continguts de la zona en la que hi té autoritat. Pot delegar a altres
    servidors DNS sub-zones d'autoritat.

    N'hi ha de dos tipus: primaris i secundaris. Els secundaris tenen la mateixa
    informació que els primaris i es fan servir per reduncància.
  \item Local: (\textit{local name server}): aquest servidor consulta iterativament
    als servidors autoritatius fins a obtenir el resultat de la consulta. A més, pot
    guardar en \textit{caché} les respostes de les consultes.
\end{itemize}

Tot i que és possible fer-ho, en general no hi ha servidors autoritatius
i locals alhora. Si es fa una consulta iterativa a un servidor autoritatiu aquest
no cercarà la resposta.

En aquesta pràctica configurarem un servidor autoritatiu de tercers. Si teniu
interès en desplegar el vostre propi servidor de noms us recomanem utilitzar
el servei \texttt{bind}.

\subsection{Dominis de primer nivell}

Els dominis de primer nivell o \textit{Top-Level Domains} (TLD) son els dominis
directament descendents de l'arrel o \textit{root}. Inicialment només n'hi havien
6: \textit{.gov}, \textit{.mil}, \textit{.net}, \textit{.edu}, \textit{.org} i \textit{.com}.

Amb el creixement d'Internet, aquest va arribar a territoris estrangers, i es va acabar
assignant un TLD a cadascun d'ells: \textit{.es}, \textit{.pt}, etc. Actualment,
l'ICANN crea TLDs per molts col·lectius diferents, i delega la seva
autoritat a una entitat. En el cas del TLD \textit{.cat}, aquest és gestionat des
dels seus inicis per la Fundació .cat : \url{https://fundació.cat}.

Quan es parla col·loquialment d'\textit{adquirir un domini}, es refereix
a arribar a un tracte amb una entitat que disposa d'un TLD perquè aquesta
ens reservi i delegui l'autoritat d'un subdomini. Cada TLD pot posar les
condicions (preu, finalitat, idioma dels continguts) que desitja. En el cas
d'un domini .cat, es demana que el contingut estigui en català.

Per facilitar la tasca de "llogar" dominis s'ha creat el concepte de
registrador de dominis o \textit{registrar} en anglès. Cada TLD escull quins
registradors poden assignar sub-dominis en nom seu, i sota quines condicions.
En el cas del domini .cat, podeu consultar els registradors disponibles a
\url{https://domini.cat/comparativa-de-preus/}.

\subsection{Registrar un domini}

La primera tasca d'aquesta pràctica consistirà en regsitrar un domini
amb un registrador. Qualsevol registrador és igual de vàlid, però si busqueu
una alternativa econòmica us recomanem \url{https://dinahosting.com} pels
dominis .cat i \url{https://cloudflare.com} per la resta (\textit{.com}, \textit{.net}, \textit{.es}, ...).

Existeixen alternatives per si no voleu pagar per un domini. Hi ha serveis com
\url{https://dynu.com} que et cedeixen gratüitament un domini de l'estil
\texttt{domini.dynu.com}. Utilitzar subdominis comporta algunes conseqüències,
però no tenen cap impacte en les pràctiques d'aquesta assignatura.

\begin{quote}
  TASCA 1: Adquiriu un domini o subdomini.
\end{quote}

\subsection{Gestionar un domini}

Cada registrador té la seva interfície, però gairebé tots permeten gestionar
els registres DNS. En cas que no ho permeti o que es vulgui utilitzar un altre
servidor de noms de domini (ja sigui propi o un altre servei), un registrador
sempre ha d'oferir la possiblitat de canviar les entrades \texttt{NS}. D'aquesta
manera és possible gestionar un domini .cat amb \textit{cloudflare}.

Un registrador també està obligat a oferir la possiblitat de traspassar el domini
a un altre registrador. És un procés lent, però que garanteix que en cap moment el
domini estarà "alliberat", és a dir, que cap persona externa el podrà adquirir.

\begin{quote}
  TASCA 2: Creeu en el gestor del registrador un registre \texttt{A} que relacioni
  el domini adquirit (o un subdomini d'aquest) a l'adreça IPv4 del VPS de la
  pràctica 1. Creeu un registre \texttt{AAAA} amb la mateixa finalitat que l'anterior
  però per l'adreça IPv6.

  Comproveu el funcionament de la tasca anterior utilitzant les eines \texttt{ping}
  o \texttt{dig} en el vostre ordinador.
\end{quote}

\subsection{DNSSEC}



Explicar:
\begin{itemize}
  \item DNS, zones d'autoritat
  \item TLD, subdomini
  \item Qui assigna TLD i com es gestionen aquests
  \item serveis de tercers
  \item Seguretat: DNSSEC
  \item DDNS per IPs variants, i altres aternatives
  \item registres PTR
\end{itemize}

\section{Desenvolupament de la pràctica}

Detallar tasques per:
\begin{itemize}
  %\item Adquirir un domini (proposar Cloudflare o dinahosting si és .cat),
  %o bé un subdomini amb serveis de tercers com dynu.com .
  %\item Posar registre A i AAAA d'un subdomini al VPS anterior.
  \item Habilitar DNSSEC i comprovar que funciona.
  \item Utilitzar dynu per posar DDNS al portàtil via cronjob. Així sabrem
  sempre a on està el nostre portàtil.
  \item Si el proveïdor ho permet, posar registre PTR.
  \item Canviar el hostname del servidor.
\end{itemize}

\section{Avaluació}
\subsection{Entrega}
\label{sec:entrega}

S'ha d'entregar un fitxer de text (\texttt{P3\_GX.txt}, on X és el número o 
lletra del grup) amb el següent format:

\begin{verbatim}
Grup: GX
Domini adquirit: asi.itic.cat
Servidor: srv1.asi.itic.cat
Contrasenya usuari profe: Lkv5YYyON8X
Port SSH: 22

(Comentaris referents a l'entrega, si s'escau)
\end{verbatim}

La majoria de dades son referents a la pràctica 1. Es demana
que les torneu a escriure, així teniu la possibiltat de canviar-les.
Fixeu-vos també que no cal introduir adreces IP, ja que es poden
obtenir a partir dels dominis.

Al servidor, creeu el directori \texttt{/entregues} a l'arrel, i a dins un
directori \texttt{/entregues/p3}. Allà hi heu de deixar els scripts i/o manuals
(el que cregueu convenient) que us farien falta si mai haguéssiu de repetir
aquesta pràctica.

Per exemple, podríeu indicar:
\begin{itemize}
  \item El proveïdor que heu utilitzat, els seus preus i ofertes, i el perquè l'heu escollit.
  \item Solucions a problemes que us heu trobat, i si cal, enllaços a fòrums o manuals.
  \item Comandes que heu utilitzat amb una petita explicació de què fan (1 línia sol ser suficient).
  \item Altres coses que hagueu fet (per exemple, configurar el tallafocs de \textit{Hetzner}).
\end{itemize}

No ha de ser extens; aquests documents us serviran a vosaltres si mai heu de
crear un nou domini.

\subsection{Qualificació}

Aquesta pràctica s'avaluarà de la següent manera. La puntuació màxima és 100.

\begin{center}%{|p{1cm}|p{3cm}|}
  \begin{tabular}{p{0.7\linewidth} l}
  \hline
  Concepte & Rang \\ \hline
  Fitxer de l'entrega (\texttt{P3\_GX.txt}) amb format correcte & $[-20, 0]$ \\
  Es pot resoldre el nom del servidor per IPv4 i IPv6 & $[0, 20]$ \\
  El hostname del servidor és el del domini & $[0, 10]$ \\
  La resolució inversa de les adreces IPv4 i IPv6 del servidor permeten obtenir el domini (FQDN) & $[0, 20]$ \\
  Qualitat dels scripts/tutorials de \texttt{/entregues/p3} & $[0,50]$ \\
  El servidor presenta problemes de seguretat greus & $[-10,0]$ \\
  \hline
  \end{tabular}
\end{center}

Nota: si el vostre proveïdor no us permet tenir una adreça IPv6 pública o
modificar els registres de la resolució inversa indiqueu-ho al fitxer de la
entrega.

Tingueu present que si algun element de la taula anterior no es pot avaluar
aquest es qualificarà amb la nota més baixa.

Si es detecta algun tipus de frau en l'entrega aquesta rebrà una puntuació de zero.

\end{document}
