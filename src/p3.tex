\documentclass{practicaitic}
\usepackage{parskip}
\usepackage{graphicx}

\usepackage[onelevel]{exercici}
%\usepackage[utf8]{inputenc}

\numpract{3}
\title{Pràctica 3: Servei de noms (DNS)}

\assignatura{Aplicacions i Serveis sobre Internet}
\autor{Eric Roy Almonacid \and Francisco del Águila López}

\begin{document}

\section{Introducció}

Per poder realitzar aquesta pràctica és indispensable haver completat la
pràctica 1, donat que treballarem sobre el VPS creat anteriorment.

\subsection{Objectius}

Al finalitzar aquesta pràctica, l'estudiantat haurà:
\begin{enumerate}
  \item Entès el funcionament de la resolució de noms de domini.
  \item Adquirit un domini o subdomini, en el qual en té autoritat.
  \item Creat registres al domini utilitzant un servei de tercers.
  \item Entès el funcionament i configurat DNSSEC al domini.
  \item Entès i configurat DDNS utilitzant serveis de tercers.
\end{enumerate}

\subsection{Condicions}

Aquesta pràctica està ca\l.librada per ésser treballada en equips de 2 persones,
i té una durada de 1 a 2 setmanes.

\subsection{Lliuraments}

S'haurà de realitzar una entrega a Atenea i mantenir operatiu el servidor 
i domini creats fins que la pràctica sigui avaluada. El format de la entrega està detallat a
l'apartat \ref{sec:entrega}.

\section{El Servei de Noms de Domini}

El Servei de noms de domini o \textit{Domain Name Server} (DNS) va ser introduït
l'any 1983 amb dues finalitats:

\begin{itemize}
  \item Poder resoldre un nom de domini a una adreça IP, per evitar haver de
  memoritzar-la.
  \item Establir l'arbre de noms: definir l'estructura jeràrquica de noms distribuida
  en zones d'autoritat. Per exemple, qualsevol domini que acabi amb \texttt{upc.edu}
  significarà que pertany a la UPC. Qualsevol domini que acabi amb \texttt{epsem.upc.edu}
  significarà que pertany a l'escola de Manresa.
\end{itemize}

Així doncs, un servidor de DNS es pot distingir segons la funció que realitza:

\begin{itemize}
  \item Autoritatiu: (\textit{authoritative name server}): forma part de l'arbre
    de noms de domini, i defineix
    els continguts de la zona en la que hi té autoritat. Pot delegar a altres
    servidors DNS sub-zones d'autoritat.

    N'hi ha de dos tipus: primaris i secundaris. Els secundaris tenen la mateixa
    informació que els primaris i es fan servir per redundància.
  \item Local: (\textit{local name server}): aquest servidor consulta iterativament
    als servidors autoritatius fins a obtenir el resultat de la consulta. A més, pot
    guardar en \textit{caché} les respostes de les consultes.
\end{itemize}

Tot i que és possible fer-ho, en general no hi ha servidors autoritatius
i locals alhora. Si es fa una consulta iterativa a un servidor autoritatiu aquest
no cercarà la resposta.

En aquesta pràctica configurarem un servidor autoritatiu de tercers. Si teniu
interès en desplegar el vostre propi servidor de noms us recomanem utilitzar
el servei \texttt{bind}.

\subsection{Dominis de primer nivell}

Els dominis de primer nivell o \textit{Top-Level Domains} (TLD) son els dominis
directament descendents de l'arrel o \textit{root}. Inicialment només n'hi havien
6: \textit{.gov}, \textit{.mil}, \textit{.net}, \textit{.edu}, \textit{.org} i \textit{.com}.

Amb el creixement d'Internet, aquest va arribar a territoris estrangers, i es va acabar
assignant un TLD a cadascun d'ells: \textit{.es}, \textit{.pt}, etc. Actualment,
l'ICANN crea TLDs per molts col·lectius diferents, i delega la seva
autoritat a una entitat. En el cas del TLD \textit{.cat}, aquest és gestionat des
dels seus inicis per la Fundació .cat : \url{https://fundacio.cat}.

Quan es parla co\l.loquialment d'\textit{adquirir un domini}, es refereix
a arribar a un tracte amb una entitat que disposa d'un TLD perquè aquesta
ens reservi i delegui l'autoritat d'un subdomini. Cada TLD pot posar les
condicions (preu, finalitat, idioma dels continguts) que desitja. En el cas
d'un domini .cat, es demana que el contingut estigui en català.

Per facilitar la tasca de "llogar" dominis s'ha creat el concepte de
registrador de dominis o \textit{registrar} en anglès. Cada TLD escull quins
registradors poden assignar sub-dominis en nom seu, i sota quines condicions.
En el cas del domini .cat, podeu consultar els registradors disponibles a
\url{https://domini.cat/comparativa-de-preus/}.

\subsection{Registrar un domini}

La primera tasca d'aquesta pràctica consistirà en registrar un domini
amb un registrador. Qualsevol registrador és igual de vàlid, però si busqueu
una alternativa econòmica us recomanem \url{https://dinahosting.com} pels
dominis .cat i \url{https://cloudflare.com} per la resta (\textit{.com}, \textit{.net}, \textit{.es}, ...).

Existeixen alternatives per si no voleu pagar per un domini. Hi ha serveis com
\url{https://dynu.com} que et cedeixen gratuïtament un domini de l'estil
\texttt{domini.dynu.com}. Utilitzar subdominis comporta algunes conseqüències,
però no tenen cap impacte en les pràctiques d'aquesta assignatura.

\begin{tasca}
  Adquiriu un domini o subdomini.
\end{tasca}

\subsection{Gestionar un domini}

Cada registrador té la seva interfície, però gairebé tots permeten gestionar
els registres DNS. En cas que no ho permeti o que es vulgui utilitzar un altre
servidor de noms de domini (ja sigui propi o un altre servei), un registrador
sempre ha d'oferir la possibilitat de canviar les entrades \texttt{NS}. D'aquesta
manera és possible gestionar un domini .cat amb \textit{cloudflare}.

Un registrador també està obligat a oferir la possibilitat de traspassar el domini
a un altre registrador. És un procés lent, però que garanteix que en cap moment el
domini estarà "alliberat", és a dir, que cap persona externa el podrà adquirir.

\begin{tasca}
  Creeu en el gestor del registrador un registre \texttt{A} que relacioni
  el domini adquirit (o un subdomini d'aquest) a l'adreça IPv4 del VPS de la
  pràctica 1. Creeu un registre \texttt{AAAA} amb la mateixa finalitat que l'anterior
  però per l'adreça IPv6.

  Comproveu el funcionament de la tasca utilitzant les eines \texttt{ping}
  o \texttt{dig} en el vostre ordinador.
\end{tasca}

\subsection{DNSSEC}

Com molts protocols creats al principi d'Internet, no es va tenir en compte
la seguretat en el DNS. Al tractar-se d'un servei que no requereix que es
verifiqui l'autenticitat de les respostes a les consultes, resulta molt
senzill realitzar atacs com \textit{DNS Poisoning}.

Per mitigar aquest atac es va desenvolupar DNSSEC, unes extensions al
protocol base que utilitzen claus asimètriques per demostrar que una
resposta prové del remitent original.
Podeu trobar més informació sobre l'atac i la mitigació d'aquest a
\url{https://www.cloudflare.com/learning/dns/dns-cache-poisoning/}.

S'entrarà més en detall amb aquest tema quan es faci una introducció a la
seguretat en les sessions de teoria. Tanmateix, al estar treballant en un
entorn hostil, es recomana utilitzar DNSSEC des del primer moment.

\begin{tasca}
  Habiliteu DNSSEC pel vostre domini. Si el gestor del DNS és el mateix
  que el registrador, aquesta tasca consistirà en fer un únic clic.
\end{tasca}

\section{DNS Invers}
% tasca: canviar el hostname per tenir FQDN

El DNS Invers o \textit{Reverse DNS} és un sistema que permet obtenir un
domini a partir d'una adreça. Moltes eines de xarxa (tcpdump, wireshark, ...)
utilitzen aquest sistema per oferir una interfície més amable a l'usuari.

Es poden fer consultes de DNS invers amb l'opció \verb|-x|.
Per exemple: \verb|dig -x 8.8.8.8| mostrarà el nom del servidor associat
a aquesta IP: \verb|dns.google|. Si fem \verb|dig dns.google| tornem a obtenir
l'adreça anterior, així tancant el cicle.

El concepte de \textit{Fully Qualified Domain Name} (FQDN) no és més
que el nom de domini complert d'un computador a Internet. Aquest FQDN
inclou el nom de l'equip i el domini al que pertany. La comanda
\verb|hostname| i les seves variants (consulteu el manual) determinen
el nom de l'equip, així com el domini associat al computador. Les
bones pràctiques associades a la gestió de sistemes (computadors)
recomanen que hagi coincidència entre el que existeix en el registres
del sistema de DNS i el que el computador té configurat tant de nom de
màquina com de nom de domini. D'aquesta manera, les consultes internes
per part del sistema operatiu coincidiran amb les consultes al servei
de DNS. Disposar d'un FQDN és de gran ajuda per donar credibilitat a
un servidor de cara a realitzar certes tasques, com per exemple enviar
correus electrònics.

\begin{tasca}
  Configureu els registres PTR al proveïdor d'Internet del vostre servidor per
  obtenir un FQDN. Si és possible, realitzeu això per l'adreça IPv4 i IPv6.
  En el cas de Hetzner, haureu d'anar a la pestanya de \textit{Networking}.

  Compte! No tots els proveïdors permeten modificar els registres PTR.

  Si esteu treballant sense VPS, el vostre proveïdor d'Internet és qui
  us hauria de proporcionar la possibilitat de definir el registre
  PTR. Pràcticament la totalitat dels proveïdors d'Internet
  \textbf{no} ofereixen aquesta opció.
\end{tasca}

Finalment, per acabar de preparar-nos per les següents pràctiques,
canviarem el nom de la màquina VPS pel FQDN que haurem creat. Els
servidors de correu utilitzen per defecte el domini que s'obté a
l'executar la comanda \verb|hostname --fqdn|. Per exemple, si el nom
de la màquina és \verb|correu| i el nom de domini és \verb|itic.cat|,
podrem crear adreces que acabin amb \texttt{@correu.itic.cat}. Per
gestionar tant nom de l'equip com domini que té internament un
computador, disposeu de les comandes \verb|hostname| amb les seves
opcions i \verb|dnsdomainname|.

\begin{tasca}
  Configureu adequadament el FQDN de la màquina virtual. Reinicia
  la màquina per aplicar els canvis.
\end{tasca}

\section{Ampliació: DNS Dinamic}

L'escassetat d'adreces IPv4 obliga els proveïdors d'internet fer alguns
malabarismes per mantenir tots els usuaris connectats. La pràctica més
utilitzada, si més no fins abans l'arribada del CG-NAT, era assignar una
adreça IPv4 a l'usuari un cop aquest connectés l'encaminador a la xarxa, i
alliberar l'adreça en el moment de la desconnexió. Aquesta tècnica es coneix
com a IP Dinàmica.

La principal desavantatge de tenir una adreça que va canviant cada cert temps és
que no ens la podem memoritzar o publicar, per tant no podem fer que altra gent
es connecti de manera fiable al nostre encaminador. Una forma popular de
so\lgem ucionar això és utilitzant un DNS Dinamic (DDNS).

El DDNS està format per dues parts:
\begin{itemize}
  \item Un servidor DNS equipat amb una API per modificar els registres.
  \item Un programa que s'executa cada pocs minuts en una màquina dintre de
  la xarxa que té una IP Dinàmica. Aquest programa esbrinarà quina és
  l'adreça IP pública actual i, mitjançant l'API anterior, actualitzarà
  el registre DNS.
\end{itemize}

Hi ha moltes combinacions d'APIs i programes que podeu utilitzar per
realitzar un DDNS. En aquesta pràctica us proposem la següent configuració.

\begin{tasca}
  Configureu un DDNS en el vostre portàtil. No servirà de gran cosa: només per
  saber des de quina IPv4 es connecta l'ordinador. A continuació teniu
  un exemple de passos a seguir, però podeu utilitzar els serveis que vulgueu:

  \begin{itemize}
    \item Creeu-vos un compte d'usuari a \url{https://dynu.com}.
    \item Delegueu un subdomini als servidors de noms de Dynu, i associeu-lo
    al vostre compte (\textit{Control Panel $\to$ DDNS Services $\to$ Add}).
    \item Creeu un \textit{shell script} seguint el tutorial de
    \url{https://www.dynu.com/DynamicDNS/IP-Update-Protocol} per poder
    actualitzar l'adreça IP del subdomini.
    \item Utilitzeu \textit{cron} o similars per fer que s'executi el script
    anterior cada cert temps. Podeu trobar un bon tutorial de \textit{crontab}
    a \url{https://cronitor.io/guides/cron-jobs}.
  \end{itemize}
\end{tasca}


\section{Ampliació: Delegació d'una zona d'autoritat}

La darrera part d'aquesta pràctica consisteix en delegar un subdomini (o zona)
al vostre VPS. En aquest insta\lgem arem un servidor de noms de domini
(com per exemple \texttt{bind}) i servirem des d'aquest les respostes
a les consultes autoritatives.

Per realitzar aquesta ampliació resulta imprescindible llegir l'enunciat
alternatiu de la pràctica 3.

\begin{tasca}
  Delegueu una zona d'autoritat al vostre VPS i serviu des d'allà algun
  registre DNS. Comproveu amb \verb|dig| que podeu fer consultes
  autoritatives però no recursives al vostre servidor.

  Per exemple, si el domini adquirit és \texttt{itic.cat}, es pot delegar
  \texttt{serv.itic.cat} al VPS amb aquestes dues línies en el vostre gestor
  de DNS:

  \begin{verbatim}
    vps.itic.cat.    A    100.200.100.200
    serv.itic.cat.   NS   vps.itic.cat
  \end{verbatim}

  Llavors, en el VPS, configureu BIND per afegir alguna cosa:

  \begin{verbatim}
    test.serv.itic.cat.    A    250.250.250.250
  \end{verbatim}
\end{tasca}

%Explicar:
%\begin{itemize}
  %\item DNS, zones d'autoritat
  %\item TLD, subdomini
  %\item Qui assigna TLD i com es gestionen aquests
  %\item serveis de tercers
  %\item Seguretat: DNSSEC
  %\item DDNS per IPs variants, i altres aternatives
  %\item registres PTR
%\end{itemize}

%\section{Desenvolupament de la pràctica}

%Detallar tasques per:
%\begin{itemize}
  %\item Adquirir un domini (proposar Cloudflare o dinahosting si és .cat),
  %o bé un subdomini amb serveis de tercers com dynu.com .
  %\item Posar registre A i AAAA d'un subdomini al VPS anterior.
  %\item Habilitar DNSSEC i comprovar que funciona.
  %\item Utilitzar dynu per posar DDNS al portàtil via cronjob. Així sabrem
  %sempre a on està el nostre portàtil.
  %\item Si el proveïdor ho permet, posar registre PTR.
  %\item Canviar el hostname del servidor.
%\end{itemize}

\section{Avaluació}
\subsection{Entrega}
\label{sec:entrega}

S'ha d'entregar un fitxer comprimit (\texttt{P3\_GX.zip}, on X és el número o 
lletra del grup) amb un fitxer de text amb el següent format:

\begin{verbatim}
Grup: GX
Domini adquirit: asi.itic.cat
VPS: srv1.asi.itic.cat
Contrasenya usuari profe: Lkv5YYyON8X
Port SSH: 22

(ometre si no heu fet les ampliacions)
DDNS: portatil.asi.itic.cat

(Comentaris referents a l'entrega, si s'escau)
\end{verbatim}

La majoria de dades son referents a la pràctica 1. Es demana
que les torneu a escriure, així teniu la possibiltat de canviar-les.
Fixeu-vos també que no cal introduir adreces IP, ja que es poden
obtenir a partir dels dominis.

Al servidor, creeu el directori \texttt{/entregues} a l'arrel, i a dins un
directori \texttt{/entregues/p3}. Allà hi heu de deixar els scripts i/o manuals
(el que cregueu convenient) que us farien falta si mai haguéssiu de repetir
aquesta pràctica. Afegiu aquests fitxers també al fitxer comprimit de l'entrega.

Per exemple, podríeu indicar:
\begin{itemize}
  \item Quin proveïdor de dominis heu utilitzat i per què.
  \item Quins subdominis heu fet servir per cada punt de la pràctica.
  \item Quins scripts o fitxers heu utilitzat o modificat.
\end{itemize}

No ha de ser extens; aquests documents us serviran a vosaltres si mai heu de
crear un nou domini.

\subsection{Qualificació}

Aquesta pràctica s'avaluarà de la següent manera. La puntuació màxima és 100.

\begin{center}%{|p{1cm}|p{3cm}|}
  \begin{tabular}{p{0.7\linewidth} l}
  \hline
  Concepte & Rang \\ \hline
  Fitxer de l'entrega (\texttt{P3\_GX.zip}) amb format correcte & $[-20, 0]$ \\
  Es pot resoldre el nom del servidor per IPv4 i IPv6 & $[0, 20]$ \\
  El hostname del servidor és el del domini & $[0, 10]$ \\
  La resolució inversa de les adreces IPv4 i IPv6 del servidor permeten obtenir el domini (FQDN) & $[0, 10]$ \\
  Correcte funcionament del DDNS & $[0, 20]$ \\
  Configuració i correcte funcionament del servidor autoritatiu muntat al VPS & $[0, 10]$ \\
  Qualitat dels scripts/tutorials de \texttt{/entregues/p3} & $[0,30]$ \\
  El servidor presenta problemes de seguretat greus & $[-10,0]$ \\
  \hline
  \end{tabular}
\end{center}

Nota: si el vostre proveïdor no us permet tenir una adreça IPv6 pública o
modificar els registres de la resolució inversa indiqueu-ho al fitxer de la
entrega.

Tingueu present que si algun element de la taula anterior no es pot avaluar
aquest es qualificarà amb la nota més baixa.

Si es detecta algun tipus de frau en l'entrega aquesta rebrà una puntuació de zero.

\end{document}
